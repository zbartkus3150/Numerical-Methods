\documentclass[12pt]{article}
\usepackage{amsmath}
\usepackage{amssymb}
\usepackage{amsfonts}
\usepackage[polish]{babel}
\usepackage[T1]{fontenc}
\usepackage[dvips]{graphicx}
\usepackage[cp1250]{inputenc}
\usepackage{listings}

\textheight 23.2 cm

\textwidth 6.0 in

\hoffset = -0.5 in

\voffset = -2.4 cm

\hyphenation{}

\frenchspacing

\begin{document}

%\thispagestyle{empty}

\vspace*{3ex}
\begin{flushright}
{\large 14 April 2019}
\end{flushright}

\begin{flushleft}
{\large Bartek \.Zy\l{}a\\
Group 2}
\end{flushleft}

\hskip3cm

\begin{center}

\Large {\bf LU-decomposition of a matrix \\ using Doolittle's method \\ in solving system of linear equations $Ax=b$.}

\vskip2ex

{\large Project 18}

\end{center}

\pagebreak

\begin{center}
\section{Method description}
\end{center}

\noindent In numerical analysis and linear algebra, LU decomposition factors a matrix as the product of a lower triangular matrix and an upper triangular matrix.

\bigskip An LU factorization refers to the factorization of matrix A, with proper row and/or column orderings or permutations, into two factors, a lower triangular matrix L and an upper triangular matrix U:
\begin{equation*}
A = LU
\end{equation*} 

Doolittle's method provides a way to factor matrix A into an LU decomposition without using Gaussian Elimination.

For a general $n\times n$ matrix A, we assume that LU decomposition exists, and write the form of L and U explicitly. We then systematically solve for the entries in L and U from the equations that result from the multiplications necessary for $A=LU$

\[
\forall i \in \{1,2,...,n-1\}
\]

\begin{equation*}
U_{i,k} = A_{i,k} - \sum_{j=0}^i(L_{i,j}\cdot U_{j,k})
\end{equation*}

\medskip for k=i,i+1,...,n-1 produces the k\textsuperscript{th} row of U.

\begin{equation*}
L_{i,k} = \frac{A_{i,k} - \sum_{j=0}^i(L_{i,j}\cdot U_{j,k})}{U_{k,k}}
\end{equation*}

\medskip for i=k,k+1,...,n-1 and L\textsubscript{i,i} = 1 produces the k\textsuperscript{th} row of L.

\vskip 5pt

\noindent
Thanks to this we can solve given system of equations $Ax=b$ more efficiently by solving 2 simpler systems, namely:
\medskip
\begin{enumerate}
\item{$LY=b$ through forward substitution} 
\item{$Ux=Y$ via back substitution}
\end{enumerate}

\pagebreak

\begin{center}
\section{Program description}
\end{center}

\noindent After You ran the program, You will see the Menu:

\vskip3ex
\begin{center}
\hspace*{-2ex}
\begin{tabular}{|c|} \hline
\\
{\bf Menu}
\\\\  Input matrix A \\\\ Input vector b \\\\
Display variables \\\\ Calculate determinant of A \\\\
Compute Ax=b \\\\  Calculate Errors \\\\
FINISH
\\\\
\hline
\end{tabular}
\end{center}

\vskip3ex

\begin{itemize}
\item If You press button ,,Input matrix A'' or ,,Input vactor b'', You will be able to input Your own matrix or vector respectively. If You do not - the default arguments will be used.
\item Option ,,Display variables'' will simply display previously input variables.
\item After clicking ,,Calculate determinant of A'' the matrix determinant using LU-decomposition will be calculated.
\item Option ,,Compute Ax=b'' will solve the system of linear equations $Ax=b$ by Doolittle's algorythm for LU-decomposition.
\item ,,Calculate Errors'' will ask for one of 3 pre-defined input for error calculation.
\item At last ,,FINISH'' will end the program.
\end{itemize}

\vskip 5pt

\noindent MATLAB functions used:
\begin{enumerate}
\item Menu\_Doolittle.m - script for graphic interface for the rest of the functions.
\item Doolittle.m - function strictly for computing LU-decomposition from given matrix A using formulas described at the beggining.
\item DoolittleErrors.m - function calculating 3 types of errors for one of 3 selected example.
\item Lower\_triangular1.m \& Upper\_triangular1.m - functions from laboratories used for computing solution for Ax=b where A is lower and upper triangular matrix respectively.

\noindent ({\bf Note.} \emph{All source codes can be found in section 5.})
\end{enumerate}


\vskip20pt

\begin{center}
\section{Numerical tests}
\end{center}

\noindent
All the numerical tests are included in the DoolittleErrors.m file. It works by implementing a simple menu in which the user can choose some specified example. For each test the script sets different A, b and z where A is the usual matrix and b is an vector that we pass to our $Doolittle$ function and z is the exact solution. Then the program calls $Doolittle$ function and calculates errors as well as the condition number for A.

\bigskip
\begin{enumerate}

\item Relative error

\[
\frac{\vert \vert X-Z \vert \vert}{\vert \vert Z \vert \vert}
\]

\item Forward stability error
\[
\frac{\vert \vert X-Z \vert \vert}{\vert \vert Z \vert \vert cond(A)}\textrm{,} \quad \textrm{where } cond(A)=\vert \vert A^{-1} \vert \vert \vert \vert A \vert \vert
\]

\item Backward stability error
\[
\frac{\vert \vert B-AX \vert \vert}{\vert \vert A \vert \vert \vert \vert X \vert \vert}
\]
\vskip5pt
\end{enumerate}

\textbf{Tests:}
$x$ is the result we obtain using our implemented $Doolittle$ function, whereas $z$ is the predefined solution wich was used to get vector $b$ by multiplication $b=A*z$.

\begin{enumerate}

\item $A=pascal(10)$ and $z=10*ones(10,1)$
\[
A=\begin{pmatrix}
1 & 1 & 1 & 1 & 1 & 1 & 1 & 1 & 1 & 1 \\
1 & 2 & 3 & 4 & 5 & 6 & 7 & 8 & 9 & 10 \\
1 & 3 & 6 & 10 & 15 & 21 & 28 & 36 & 45 & 55 \\
1 & 4 & 10 & 20 & 35 & 56 & 84 & 120 & 165 & 220 \\
1 & 5 & 15 & 35 & 70 & 126 & 210 & 330 & 495 & 715 \\
1 & 6 & 21 & 56 & 126 & 252 & 462 & 792 & 1287 & 2002 \\
1 & 7 & 28 & 84 & 210 & 462 & 924 & 1716 & 3003 & 5005 \\
1 & 8 & 36 & 120 & 330 & 792 & 1716 & 3432 & 6435 & 11440 \\
1 & 9 & 45 & 165 & 495 & 1287 & 3003 & 6435 & 12870 & 24310 \\
1 & 10 & 55 & 220 & 715 & 2002 & 5005 & 11440 & 24310 & 48620
\end{pmatrix}\textbf{, }
z=\begin{pmatrix}
10 \\
10 \\
10 \\
10 \\
10 \\ 
10 \\
10 \\
10 \\
10 \\
10
\end{pmatrix}
\]
Result:
\[
x=\begin{pmatrix}
10 \\
10 \\
10 \\
10 \\
10 \\ 
10 \\
10 \\
10 \\
10 \\
10
\end{pmatrix}
\]
\begin{table}[h!]
\caption{\footnotesize 1st example $cond(A)$ \& Errors} \vskip1ex
\renewcommand{\arraystretch}{1.1}
\centering\begin{tabular}{|c|c|c|c|c|c|}
\hline $cond(A) $ & Relative & Fwd stability & Back stability\\ 
& error & error & error\\
\hline $4.1552\cdot10^{9}$ & 0 & 0 & 0\\
\hline
\end{tabular}
\end{table}

\item $A=pascal(15)$ and $z=15*ones(15,1)$
\[
A=\left(\begin{smallmatrix}
1 & 1 & 1 & 1 & 1 & 1 & 1 & 1 & 1 & 1 & 1 & 1 & 1 & 1 & 1 \\
1 & 2 & 3 & 4 & 5 & 6 & 7 & 8 & 9 & 10 & 11 & 12 & 13 & 14 & 15 \\
1 & 3 & 6 & 10 & 15 & 21 & 28 & 36 & 45 & 55 & 66 & 78 & 91 & 105 & 120 \\
1 & 4 & 10 & 20 & 35 & 56 & 84 & 120 & 165 & 220 & 286 & 364 & 455 & 560 & 680 \\
1 & 5 & 15 & 35 & 70 & 126 & 210 & 330 & 495 & 715 & 1001 & 1365 & 1820 & 2380 & 3060 \\
1 & 6 & 21 & 56 & 126 & 252 & 462 & 792 & 1287 & 2002 & 3003 & 4368 & 6188 & 8568 & 11628 \\
1 & 7 & 28 & 84 & 210 & 462 & 924 & 1716 & 3003 & 5005 & 8008 & 12376 & 18564 & 27132 & 38760 \\
1 & 8 & 36 & 120 & 330 & 792 & 1716 & 3432 & 6435 & 11440 & 19448 & 31824 & 50388 & 77520 & 116280 \\
1 & 9 & 45 & 165 & 495 & 1287 & 3003 & 6435 & 12870 & 24310 & 43758 & 75582 & 125970 & 203490 & 319770 \\
1 & 10 & 55 & 220 & 715 & 2002 & 5005 & 11440 & 24310 & 48620 & 92378 & 167960 & 293930 & 497420 & 817190 \\
1 & 11 & 66 & 286 & 1001 & 3003 & 8008 & 19448 & 43758 & 92378 & 184756 & 352716 & 646646 & 1144066 & 1961256 \\
1 & 12 & 78 & 364 & 1365 & 4368 & 12376 & 31824 & 75582 & 167960 & 352716 & 705432 & 1352078 & 2496144 & 4457400 \\
1 & 13 & 91 & 455 & 1820 & 6188 & 18564 & 50388 & 125970 & 293930 & 646646 & 1352078 & 2704156 & 5200300 & 9657700 \\
1 & 14 & 105 & 560 & 2380 & 8568 & 27132 & 77520 & 203490 & 497420 & 1144066 & 2496144 & 5200300 & 10400600 & 20058300 \\
1 & 15 & 120 & 680 & 3060 & 11628 & 38760 & 116280 & 319770 & 817190 & 1961256 & 4457400 & 9657700 & 20058300 & 40116600
\end{smallmatrix} \right)
\]
\[
z=\begin{pmatrix}
15 \\
15 \\
15 \\
15 \\
15 \\ 
15 \\
15 \\
15 \\
15 \\
15 \\
15 \\
15 \\
15 \\
15 \\
15
\end{pmatrix}
\]
Result:
\[
x=\begin{pmatrix}
15 \\
15 \\
15 \\
15 \\
15 \\ 
15 \\
15 \\
15 \\
15 \\
15 \\
15 \\
15 \\
15 \\
15 \\
15
\end{pmatrix}
\]
\begin{table}[h!]
\caption{\footnotesize 2nd example $cond(A)$ \& Errors} \vskip1ex
\renewcommand{\arraystretch}{1.1}
\centering\begin{tabular}{|c|c|c|c|c|c|}
\hline $cond(A) $ & Relative & Fwd stability & Back stability\\ 
& error & error & error\\
\hline $2.8397\cdot10^{15}$ & 0 & 0 & 0\\
\hline
\end{tabular}
\end{table}

\item $A=hild(10)$ and $z=ones(10,1)$
\[
A=\begin{pmatrix}
1.0000 & 0.5000 & 0.3333 & 0.2500 & 0.2000 & 0.1667 & 0.1429 & 0.1250 & 0.1111 & 0.1000 \\
0.5000 & 0.3333 & 0.2500 & 0.2000 & 0.1667 & 0.1429 & 0.1250 & 0.1111 & 0.1000 & 0.0909 \\
0.3333 & 0.2500 & 0.2000 & 0.1667 & 0.1429 & 0.1250 & 0.1111 & 0.1000 & 0.0909 & 0.0833 \\
0.2500 & 0.2000 & 0.1667 & 0.1429 & 0.1250 & 0.1111 & 0.1000 & 0.0909 & 0.0833 & 0.0769 \\
0.2000 & 0.1667 & 0.1429 & 0.1250 & 0.1111 & 0.1000 & 0.0909 & 0.0833 & 0.0769 & 0.0714 \\
0.1667 & 0.1429 & 0.1250 & 0.1111 & 0.1000 & 0.0909 & 0.0833 & 0.0769 & 0.0714 & 0.0667 \\
0.1429 & 0.1250 & 0.1111 & 0.1000 & 0.0909 & 0.0833 & 0.0769 & 0.0714 & 0.0667 & 0.0625 \\
0.1250 & 0.1111 & 0.1000 & 0.0909 & 0.0833 & 0.0769 & 0.0714 & 0.0667 & 0.0625 & 0.0588 \\
0.1111 & 0.1000 & 0.0909 & 0.0833 & 0.0769 & 0.0714 & 0.0667 & 0.0625 & 0.0588 & 0.0556 \\
0.1000 & 0.0909 & 0.0833 & 0.0769 & 0.0714 & 0.0667 & 0.0625 & 0.0588 & 0.0556 & 0.0526 
\end{pmatrix}
\]
\[
z=\begin{pmatrix}
1 \\
1 \\
1 \\
1 \\
1 \\ 
1 \\
1 \\
1 \\
1 \\
1 
\end{pmatrix}
\]
Result:
\[
x=\begin{pmatrix}
1.0000 \\
1.0000 \\
1.0000 \\
1.0000 \\
0.9998 \\
1.0004 \\
0.9993 \\ 
1.0007 \\
0.9996 \\
1.0001
\end{pmatrix}
\]
\begin{table}[h!]
\caption{\footnotesize 3rd example $cond(A)$ \& Errors} \vskip1ex
\renewcommand{\arraystretch}{1.1}
\centering\begin{tabular}{|c|c|c|c|c|c|}
\hline $cond(A) $ & Relative & Fwd stability & Back stability\\ 
& error & error & error\\
\hline $1.6025\cdot10^{13}$ & $3.5784\cdot10^{-4}$ & $2.2330\cdot10^{-17}$ & $7.4983\cdot10^{-17}$\\
\hline
\end{tabular}
\end{table}

\end{enumerate}

\begin{center}
\section{Conclusions}
\end{center}

\begin{flushleft}
As we can see, solving equations of type $Ax=b$, where $A \in \mathbb R^{n \times n}$ and $b \in \mathbb R^{n \times 1}$ by the Doolittle method $(A=LU)$ is an accurate mathod and does not produce big errors for matrices with $eps\cdot cond(A) < 1$ which is to be expected.
We obtained the biggest error when using Hilbert's 10 by 10 matrix, which is predictable as it is a matrix with relatively big conditional number ($cond(hilb(10))$ was in  $10^{13}$ range) and it has more complicated 1\textsuperscript{st} column then the other examples. Both Forward and Backward stability errors are small or non-existing in every case, moeover as long as the condition number was reasonably small, the relative error was also very small or non-existant. All of this together suggests that the method is quite accurate.	
\end{flushleft}

\pagebreak

\begin{center}
\item \section{Source Codes}
\end{center}


{\bf Menu\_Doolitttle.m:}
\lstinputlisting{Menu_Doolittle.m}

\medskip 
{\bf Doolittle.m:}
\lstinputlisting{Doolittle.m}

\medskip 
{\bf DoolittleErrors.m:}
\lstinputlisting{DoolittleErrors.m}

\medskip
 {\bf Lower\_triangular1.m:}
\lstinputlisting{Lower_triangular1.m}

\medskip
{\bf Upper\_triangular.m:}
\lstinputlisting{Upper_triangular1.m}

\end{document}
